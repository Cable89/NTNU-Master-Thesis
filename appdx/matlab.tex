\lstset{language=Matlab,%
    %basicstyle=\color{red},
    breaklines=true,%
    morekeywords={matlab2tikz},
    keywordstyle=\color{blue},%
    morekeywords=[2]{1}, keywordstyle=[2]{\color{black}},
    identifierstyle=\color{black},%
    stringstyle=\color{mylilas},
    commentstyle=\color{mygreen},%
    showstringspaces=false,%without this there will be a symbol in the places where there is a space
    numbers=left,%
    numberstyle={\tiny \color{black}},% size of the numbers
    numbersep=9pt, % this defines how far the numbers are from the text
    emph=[1]{for,end,break},emphstyle=[1]\color{red}, %some words to emphasise
    %emph=[2]{word1,word2}, emphstyle=[2]{style},    
}

\chapter{Matlab code}
\section{Transfer}
\begin{lstlisting}
C1 = 1e-7;
C2 = 1e-11;
L1 = 1e-5;
L2 = 1e-1;
k = 0.2;
M  = k*sqrt(L1*L2); %2e-6;
R1 = 1;
R2 = 1e2;
G1 = 2e-6;

fprintf('Expected resonance frequency:\n');
fprintf('%i Hz\n', 1./(2*pi()*sqrt(C1*L1)));

H1 = trans(C1, C2, L1, L2, M, R1, R2, G1);
%H2 = trans(C1, C2, L1, L2, M, R1, R2, G1);
%H3 = trans(C1, C2, L1, L2, M, R1, R2, G1);
%H4 = trans(C1, C2, L1, L2, M, R1, R2, G1);
%H5 = trans(C1, C2, L1, L2,  M, R1, R2, G1);

figure('Name','L2');
bde1 = bodeplot(H1);%,H2,H3,H4,H5);
setoptions(bde1, 'FreqUnits','Hz','Grid','on','Xlim',[1e4, 2e6]);
%legend('0.01','0.8','1.0','1.2','100','Location','northeast');

[mag, phase, W] = bode(H1);
[val, idx] = max(20*log10(squeeze(mag(1,1,:))));
W = W./(2*pi);
fprintf('Max amplitude:\n');
fprintf('%i dB\n', val);
fprintf('%i Hz\n', W(idx));

figure;
subplot(2,1,1)
step(H1, 2e-5); hold on;
[Y, T] = step(H1, 2e-5);
%findpeaks(Y)
[pks1, locs1] = findpeaks(-Y);
[pks2, locs2] = findpeaks(Y);
stepfrequency = 1./(2*(T(locs2(1))-T(locs1(1))))


subplot(2,1,2)
impulse(H1, 2e-5);

figure;
pzplot(H1);
grid on;
[P, Z] = pzmap(H1)

function H = trans(C1, C2, L1, L2, M, R1, R2, G1)
    a  = ((C1*C2*G1*L1*L2)-2*(C1*C2*G1*L1*M)+(C1*C2*G1*M^2));
    b  = ((C1*C2*G1*L1*R2)+(C1*C2*G1*L2*R1)-2*(C1*C2*G1*M*R1)+(C1*C2*L1));
    c  = ((C1*C2*G1*R1*R2)+(C1*C2*R1)+(C1*G1*L1)+(C2*G1*L2)-2*(C2*G1*M));
    d  = ((C1*G1*R1)+(C2*G1*R2)+C2);
    e  = (G1);
    f  = (-1)*(C1*C2*M);
    g  = (-1)*(C1*G1*M);

    H  = tf([f g 0 0],[a b c d e]);
end

function H = transferCurrent(C1, C2, L1, L2, M, R1, R2, G1)
    a  = 2*(C1*C2*G1*M)-(C1*C2*G1*L2);
    b  = 0-(C1*C2*G1*R2)-(C1*C2);
    c  = 0-(C1*G1);
    d  = 0;
    e  = (C1*C2*G1*L1*L2)-2*(C1*C2*G1*L1*M);
    f  = (C1*C2*G1*L1*R2)+(C1*C2*L1)+(C1*C2*G1*R1*L1)-2*(C1*C2*G1*R1*M);
    g  = (C1*G1*L1)+(C1*C2*R1)+(C2*G1*L2)-2*(C2*G1*M);
    h  = (C1*G1*R1)+(C2*G1*R2)+C2;
    k  = G1;

    H  = tf([a b c d],[e f g h k]);
end
\end{lstlisting}

\newpage
\section{Linear simulation}
\begin{lstlisting}
C1 = 1e-7;
C2 = 1e-11;
L1 = 1e-5;
L2 = 1e-1;
k = 0.2;
M  = k*sqrt(L1*L2); %2e-6;
R1 = 1;
R2 = 1e2;
G1 = 2e-6;
%G1 = 2e-7;

% Time domain parameters
fs = 4e6;       % Sampling frequency
dt = 1/fs;      % Time resolution
T = 1;          % Signal duration
t = 0:dt:T-dt;  % Total duration
N = length(t);  % Number of time samples


f0=1/(2*pi*(sqrt(L1*C1)))
f0_s=1/(2*pi*(sqrt(L2*C2)))
f0=1.57e+05
T0 = (fs/f0);
n = 10;

x2=square(2*pi*f0*t);
x2 = x2*160;
x2 = x2(1:int16(n*T0));
x2 = [x2 zeros(1,(N-length(x2)), 'int16')];

H  =  trans(C1, C2, L1, L2, M, R1, R2, G1);

figure;

lsim(H,x2,t);
axis([0 1.5e-4 -5e4 5e4]);
xlabel('Time [s]');
ylabel('Amplitude [V]');
pbaspect([2 1 1]);
\end{lstlisting}

\section{Limiter filter}
\begin{lstlisting}
R2 = 10;
C3 = 1e9;
n1 = 1;
n2 = 100;

H = tf([R2],[R2*C3 1])

bodeplot(H);
[mag,phase,wout] = bode(H);
\end{lstlisting}

\newpage
\section{Audio plot and synthesis}
\begin{lstlisting}
%% Time domain parameters
fs = 96000;     % Sampling frequency
dt = 1/fs;      % Time resolution
T = 5;          % Signal duration
t = 0:dt:T-dt;  % Total duration
N = length(t);  % Number of time samples

%% Signal generation
envelope = [-41 -41 -44 -40 -47 -51 -55 -55 -58 -60 -60 -64 -62 -62 -65 -62 -66 -66 -67 -70 -72 -80]; % In dB/Hz
envelope = db2mag(envelope);
%envelope = envelope+10;
f0 = 1000;       % fundamental frequency
x = envelope(1)*sin(2*pi*f0*t); % fundamental sinusoid
for i = 2:22
    x = x + envelope(i)*sin(2*pi*f0*i*t);
end

%% Magic
[y,fsy] = audioread('1khz - 09.wav');
y = y(fs*2:fs*7);

%% Generated
figure;
subplot(2,2,1);
plot(psd(spectrum.periodogram,x,'Fs',fs,'NFFT',length(x)));
axis([0 25 -100 -30]);
px = audioplayer(x, fs);
%play(px, [1 (get(px, 'SampleRate') * 3)]);
subplot(2,2,2);
plot(t(1:fs*1e-2), x(1:fs*1e-2));
title('Synthesized signal');
xlabel('Time (s)');
ylabel('Amplitude');
%pause(6);

%% Measured
subplot(2,2,3);
plot(psd(spectrum.periodogram,y,'Fs',fs,'NFFT',length(y)));
axis([0 25 -100 -30]);
py = audioplayer(y, fs);
subplot(2,2,4);
plot(t(1:fs*1e-2), y(1:fs*1e-2));
title('Recorded signal');
xlabel('Time (s)');
ylabel('Amplitude');
%play(py, [1 (get(py, 'SampleRate') * 3)]);

figure;
plot(psd(spectrum.periodogram,y,'Fs',fs,'NFFT',length(y)));
axis([0 25 -100 -30]);
pbaspect([2 1 1])
figure;
plot(t(1:fs*1e-2), y(1:fs*1e-2));
pbaspect([2 1 1]);
title('Recorded signal');
xlabel('Time (s)');
ylabel('Amplitude');
\end{lstlisting}