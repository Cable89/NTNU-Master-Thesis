\section*{Sammendrag}
Universiteter, vitenskapssentre og andre institusjoner har behov for utstyr som demonstrerer fysiske fenomener på interessante og publikumsvennlige måter. Et eksempel på dette er en Double Resonant Solid State Tesla Coil (DRSSTC). Det er ikke noe stort kommersielt eller faglig miljø for tesla-spoler, men tesla-spoler er for det meste utformet og brukt som en hobby. Det er noen få personer som deler design på internett som de fleste tesla-spoledesign er basert på. Få eller ingen av disse underbygger designvalg eller antagelser med fysikk og matte. Denne oppgaven vil forsøke å underbygge designvalgene som er gjort i driveren til en musikalsk Dual Resonant Solid State Tesla Coil.

Denne oppgaven diskuterer en implementering av en DRSSTC, underbygger funksjonaliteten til underkretser og komponenter med matematikk. Deretter diskuteres den matematiske beskrivelsen av resonanttransformatoren, komponentstørrelser varieres, overføringsfunksjonene plottes og simuleres. Deretter presenteres noen målinger gjort på en fysisk implementert DRSSTC.

Komponentverdiene eller parametrene som må tilpasses resonansfrekvensen er; forsinkelsen i sperreinnstillingsnettverket i avbryteren, faseovergangstiden i avbryteren og hjørnefrekvensen av støyfilteret i begrenseren. Komponentverdiene eller parametrene som må tilpasses til strømmen i den primære resonanskretsen er; impedansen til tilbakekoblingsbelastningen i avbryteren og i begrenseren, og antall viklinger på tilbakekoblingstransformatorene.

I resonanskretsen har vi vist at de ohmske motstandene i både primær- og sekundærkretsen bør være så små som mulig for å gi en høy som mulig amplitude på utgangen, konduktansen til gnisten ser ikke ut til å påvirke resonansfrekvensen (detuning) eller amplitude på utgangen, men påvirker strømmen i den primære resonanskretsen. Og vil påvirke tilbakekoblingssignalene. Koblingskoeffisienten påvirker bare amplituden på utgangen, så lenge det ikke oppstår lysbue mellom primær- og sekundærspolene. I henhold til overføringsfunksjonene gir ikke variasjon av kondensatorene eller spolene de samme resultatene som forventet fra de vanligste antagelsene i hobbymiljøet. Her kan det ikke trekkes andre konklusjoner enn at dette kan være et tema for videre forskning.