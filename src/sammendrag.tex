\section*{Sammendrag}
Universities, science centers and other institutions has the need for equipment that demonstrates physical phenomena in interesting and audience friendly ways. An example of this is a Double Resonant Solid State Tesla Coil (DRSSTC). There is no big commercial or academic community for tesla coils, but tesla coils are mostly designed and used as a hobby. There are a few people sharing their designs on the internet that most of the tesla coil designs are based on. Few or none of these substantiate design choices or assumptions with physics and math. This thesis will attempt to substantiate the design choices done in the driver for a musical dual resonant solid state tesla coil.

There have been done electrical and acoustic measurements on a physical implementation. The acoustic measurements may be used for further research on the streamer, and any alternative or related ways of generating an acoustic signal with a tesla coil.

A implementation of a DRSSTC has been analyzed and described. The intended function of components have been presented and substantiated with mathematics. The component values or parameters that need to be adapted to the resonant frequency are; the delay in the latch reset network $t_r$ given by $R_3$ and $C_2$ in the interrupter, the phase lead time $t_d$ given by $L_1$ and $R_2$ in the interrupter, and the corner frequency of the noise filter $f_c$ given by $R_2$ and $C_3$ in the limiter.

The component values or parameters that need to be adapted to the current flowing in the primary resonant circuit $I_1$ are; $|Z_L|$ given by $L_1$ and $R_2$ in the interrupter, $R_2$ in the limiter, and the number of turns $n$ on $L_3$ and $L_4$ in the primary resonant circuit. $|Z_L|$ affects.

In the resonant circuit we have proven that the ohmic resistances $R_1$ and $R_2$ in both the primary and secondary circuit should be as small as possible to give a high as possible amplitude on the output, the conductance of the streamer $G_1$ does not seem to affect the resonant frequency (detuning) or the amplitude on the output, but does affect the current in the primary resonant circuit. And will affect the feedback signals $X8$ and $X9$. The coupling coefficient $k$ only affects the amplitude on the output, as long as no arcing happens between the primary and secondary coils $L_1$ and $L_2$. According to the transfer functions $H(s)$ and $H_{FB}(s)$ varying $C_1$, $L_1$, $C_2$, or $L_2$ does not give the same results as expected from the common assumptions in the hobby community. Here no other conclusions can be drawn other that this may be a topic for further research.
% Abstract in norwegian