\section*{Sammendrag}
Universiteter, vitenskapssentre og andre institusjoner har behov for utstyr som demonstrerer fysiske fenomener på interessante og publikumsvennlige måter. Et eksempel på dette er en Double Resonant Solid State Tesla Coil (DRSSTC). Det er ikke stort kommersielt eller faglig samfunn for tesla-spoler, men tesla-spoler er for det meste utformet og brukt som en hobby. Det er noen få personer som deler design på internett som de fleste av tesla-spolens design er basert på. Få eller ingen av disse underbygger designvalg eller antagelser med fysikk og matte. Denne oppgaven vil forsøke å underbygge designvalgene som er gjort i sjåføren for en musikalsk dual resonant solid state tesla coil.

Denne oppgaven diskuterer en implementering av et DRSSTC, underbygger funksjonaliteten til underkretser og komponenter med matematikk. Deretter diskuteres den matematiske beskrivelsen av resonanttransformatoren, varierende komponentstørrelser, plotting og simulering av overføringsfunksjonene. Deretter presenteres noen målinger gjort på en fysisk implementert DRSSTC.

Komponentverdiene eller parametrene som må tilpasses resonansfrekvensen er; Forsinkelsen i sperreinnstillingsnettverket i avbryteren, faseovergangstiden i avbryteren og hjørnefrekvensen av støyfilteret i begrenseren. Komponentverdiene eller parametrene som må tilpasses til strømmen som strømmer i den primære resonanskrets er; Impedansen til tilbakekoblingsbelastningen i avbryteren og i begrenseren, og antall svinger på gjeldende tilbakekoblingstransformatorer.

I resonanskretsen har vi bevist at de ohmiske motstandene i både primær- og sekundærkretsen bør være så små som mulig for å gi en høy som mulig amplitude på utgangen, synes ikke strømmen av strømmen til å påvirke resonansfrekvensen (detuning ) Eller amplitude på utgangen, men påvirker strømmen i den primære resonanskretsen. Og vil påvirke tilbakemeldingssignalene. Koblingskoeffisienten påvirker bare amplituden på utgangen, så lenge det ikke skjer noen bue mellom primær- og sekundærspolene. I henhold til overføringsfunksjonene varierer kondensatorene eller induktørene ikke med de samme resultatene som forventet fra de felles antagelsene i hobbyområdet. Her kan det ikke trekkes andre konklusjoner om at dette kan være et tema for videre forskning.