\newpage
\section*{Preface}

Omega Verksted has a long tradition with providing Tesla Coil shows for the line union 'Omega' and for other student societies at NTNU. The current implementation is based on some inadequate documentation from the 2009 implementation. The 2009 implementation was in use up to 2014 and was notorious for blowing output transistors. As well as for being put together with hot glue. In 2014 an effort was begun to improve the reliability and portability of the tesla coil. A new implementation was created with the design split into modules to ease the further development. The connectors and the casing for the driver was replaced and a back plane architecture was selected improving portability and reliability greatly. And a project report was written on the back plane architecture \citep{prosjektoppgave}. The coil rig was also replaced as it had the tendency to catch fire. The mode of operation for this implementation of a DRSSTC or general DRSSTC implementations are still not understood fully by all members of the project. This thesis aims do provide adequate documentation on the current DRSSTC implementation at Omega Verksted. And to serve as an educational document to give project members the knowledge to change or improve on the implementation.

Thanks should be made to Omega Verksted who has financed the parts used for this project, Tim Cato Nedland at the department of acoustics who did the acoustic measurements, the department of electrical power engineering who provided a safe location to do measurements, and HSE coordinator Sverre Vegard Pettersen who helped to make it feasible to do safe measurements. Lastly thanks should be made to my supervisor Lars Lundheim who have provided invaluable support and guidance through the work on this thesis.