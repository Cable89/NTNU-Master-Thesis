\section{Earlier work}
%Who has done work on this before? steve4hv, pupman, scantesla, loneoceans.
There is no big commercial or academic community for tesla coils, but tesla coils are mostly designed and used as a hobby. There are a few people sharing their designs on the internet that most of the tesla coil designs are based on. Some of these are; Steve Ward \Citep{stevehv}, Jimmy Hynes \Citep{chunkyboy86}, Terry Fritz \Citep{terrel}, Steve Conner \Citep{conner}, Dan McCauley \Citep{easternvoltage}, and the tesla coil mailing list \Citep{pupman}.

Among these there is few disagreements on how a tesla coil should be designed, there are few variations on the resonant circuit, but slightly more variation on the low voltage controlling side. Some use a preset oscillator to drive the resonant circuit, while others use feedback from either the primary or secondary resonant circuits.

There has been developed an drsstc tesla coil driver by Steve Ward witch other people have designed variations on \Citep{ud27}.

Simulation software for the resonant circuitry of tesla coils called Scantesla have been written \Citep{scantesla}.

Guidelines for the design of the resonant circuits can be found at most of the above mentioned sites.

A modular back plane based tesla coil driver implementation have been designed and implemented by the author together with other members of Omega Verksted\footnote{Omega Verksted is a association of electronics and hobby interested students at the Norwegian University of Science and Technology (NTNU) founded in 1971.}. This is based on earlier designs by members of Omega Verksted, the last major implementation at Omega Verksted was done in 2009 mostly by Dewald de Bruyn.

Few or none of these substantiate design choices or assumptions with physics and math, this thesis will attempt to substantiate the design choices done in the driver for a musical dual resonant solid state tesla coil.

%Onetesla sells commercially

%Tables for coil aspect ratios.