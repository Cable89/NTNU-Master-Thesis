\section*{Abstract}
Universities, science centers and other institutions has the need for equipment that demonstrates physical phenomena in interesting and audience friendly ways. An example of this is a Double Resonant Solid State Tesla Coil (DRSSTC). There is no big commercial or academic community for tesla coils, but tesla coils are mostly designed and used as a hobby. There are a few people sharing their designs on the internet that most of the tesla coil designs are based on. Few or none of these substantiate design choices or assumptions with physics and math. This thesis will attempt to substantiate the design choices done in the driver for a musical dual resonant solid state tesla coil.

This thesis discusses an implementation of a DRSSTC, substantiates the functionality of sub circuits and components with mathematics. Then discusses the mathematical description of the resonant transformer, varying component sizes, plotting and simulating the transfer functions. Then presents some measurements done on a physical implemented DRSSTC.

The component values or parameters that need to be adapted to the resonant frequency are; the delay in the latch reset network in the interrupter, the phase lead time in the interrupter, and the corner frequency of the noise filter in the limiter. The component values or parameters that need to be adapted to the current flowing in the primary resonant circuit are; the impedance of the feedback load in the interrupter and in the limiter, and the number of turns on the current feedback transformers.

In the resonant circuit we have proven that the ohmic resistances in both the primary and secondary circuit should be as small as possible to give a high as possible amplitude on the output, the conductance of the streamer does not seem to affect the resonant frequency (detuning) or the amplitude on the output, but does affect the current in the primary resonant circuit. And will affect the feedback signals. The coupling coefficient only affects the amplitude on the output, as long as no arcing happens between the primary and secondary coils. According to the transfer functions varying the capacitors or inductors does not give the same results as expected from the common assumptions in the hobby community. Here no other conclusions can be drawn other that this may be a topic for further research.
