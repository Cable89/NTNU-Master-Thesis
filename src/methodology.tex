\section{Methodology}

\subsection{Acoustic measurements}

The acoustic measurements of the tesla coil were performed in the high voltage laboratory at the department of electrical power engineering at ntnu. This was the only location fitting after health and safety considerations were made. This room is not designed for acoustic measurements, and has significant ambient noise and reflections.

The coil rig were placed in the room with a grounding electrode consisting an orb and spike with identical design as the topload placed a distance of ~40cm away. The driver was placed 3 meters away connected with a 3m long cable. The signal generator, pulse shaper, and recording equipment was placed adjacent to the driver.

The recording was performed by the department of acoustics at ntnu.

Since the room was not meant for acoustic measurements the impulse response of the room was measured. This measurement was also performed by the department of acoustics at ntnu. This was done by placing a speaker approximately in the center of the room, and a microphone were the microphone was to be placed during recordings. Then a sweep was played on the speaker, and recorded.

The recordings of single tones was performed by setting the signal generator to the wanted input frequency and then the output sound was recorded.

The recordings of musical audio files was performed by playing the file from a computer and then recording the output sound.

The sweep was recorded by playing a sweep from a different computer, and then recording, When playing the sweep we discovered that the computer playing back the sweep (and recording) was affected by the electromagnetic field generated by the tesla coil, and the playback was thus choppy. It was later discovered that all recordings done were choppy.

%Department of acoustics

%Impulse response of room measured first

%Recording is lagging, affected by electromagnetic field

%Recording of different (single) tones

%Recording of musical audio files

%Recording of sweep