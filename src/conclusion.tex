\chapter{Conclusion}
An implementation of a DRSSTC has been analyzed and described. The intended function of components have been presented and substantiated with mathematics. The component values or parameters that need to be adapted to the resonant frequency are; the delay in the latch reset network $t_r$ given by $R_3$ and $C_2$ in the interrupter, the phase lead time $t_d$ given by $L_1$ and $R_2$ in the interrupter, and the corner frequency of the noise filter $f_c$ given by $R_2$ and $C_3$ in the limiter. $t_r$ affects the acoustic signal if the synchronous shutdown in the interrupter does not function correctly. $t_d$ affects the heat generated in the power amplifier, and the amplitude on the output. $f_c$ can affect noise on the acoustic signal.

The component values or parameters that need to be adapted to the current flowing in the primary resonant circuit $I_1$ are; $|Z_L|$ given by $L_1$ and $R_2$ in the interrupter, $R_2$ in the limiter, and the number of turns $n$ on $L_3$ and $L_4$ in the primary resonant circuit. $|Z_L|$ affects. $R_2$ affects the range of amplitude attainable on the output. $n$ affects the selection of $|Z_L|$ and $R_2$.

In the resonant circuit we have shown that the ohmic resistances $R_1$ and $R_2$ in both the primary and secondary circuit should be as small as possible to give a high as possible amplitude on the output, the conductance of the streamer $G_1$ does not seem to affect the resonant frequency (detuning) or the amplitude on the output, but does affect the current in the primary resonant circuit. And will affect the feedback signals $X8$ and $X9$. The coupling coefficient $k$ only affects the amplitude on the output, as long as no arcing happens between the primary and secondary coils $L_1$ and $L_2$. According to the transfer functions $H(s)$ and $H_{FB}(s)$ varying $C_1$, $L_1$, $C_2$, or $L_2$ does not give the same results as expected from the common assumptions in the hobby community. Here no other conclusions can be drawn other that this may be a topic for further research.

The electrical measurements done on the physical implementation further substantiate the mode of operation of the driver. The acoustic measurements substantiate that an acoustic signal can be generated with the circuits presented here, and may be used for further research on the streamer and any alternative or related ways of generating an acoustic signal with a tesla coil. This report may also give an operator insight into what input signals $X1$ are suitable for this implementation of driver.


% Resonansfrekvens
%TK514 R3,C2 -> T
%TK514 L1, R2 -> t_d
%TK513 R2, C3 -> f_c

% Strøm
%TK514 L1, R2 -> |Z_L| feedbackspenning
%TK513 R2 -> feedbackspenning
% n1/n2 -> feedbackspenning


% Resonanskrets
