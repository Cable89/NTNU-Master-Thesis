\section{Signals in the DRSSTC}
A description of the different signals in the DRSSTC

\subsection{Triggering signal (X2)}
\label{triggering_signal}
The triggering signal $X2$ is two level and contains two pieces of information from $X1$, the frequency $f$ (tone) and the volume (intensity). The frequency is given by the time $T$ between the positive flanks of the signal $f=\frac{1}{T}$. This is the base harmonic of the acoustic tone heard at the output of the system. The volume is given by the duty cycle of the pulses (the relationship between the pulse being high and the total period $T$ of the pulse. \Cref{fig:tones} shows different tones, and \cref{fig:volumes} shows different volumes.

\begin{figure}[!ht]
    \centering
    \begin{tikztimingtable}
        X2 & 2L 1H 7L 1H 7L 1H 7L 1H 7L\\
        X2 & 2L 1H 5L 1H 5L 1H 5L 1H 5L 1H 5L 1H 1L\\
    \end{tikztimingtable}
    \caption{Different tones}
    \label{fig:tones}
\end{figure}{}

\begin{figure}[!ht]
    \centering
    \begin{tikztimingtable}
        X2 & 2L 1H 7L 1H 7L 1H 7L 1H 7L\\
        X2 & 2L 3H 5L 3H 5L 3H 5L 3H 5L\\
    \end{tikztimingtable}
    \caption{Different volumes}
    \label{fig:volumes}
\end{figure}{}

\subsection{Feedback signal (X8)}
The feedback signal X8 and X9, are coupled with a feedback transformer to the primary resonant circuit. The feedback transformers are hand wound 100 turns around a ferrite toroid wich the wire in the resonant circuit is fed trough. If this transformer is loaded with a transistor we will get a voltage over the transistor dependent on and in phase with the current running in the primary resonance circuit. The current running 

\subsection{Primary resonant circuit signal (X6)}

\subsection{Secondary resonant circuit signal (X7)}